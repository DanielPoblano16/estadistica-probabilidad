\documentclass[11pt]{report}
\usepackage{bm}
\usepackage[utf8]{inputenc}
\usepackage[spanish]{babel}
\usepackage{dirtytalk}
\usepackage[none]{hyphenat} %Para evitar el corte de palabras
\usepackage{amsmath}
\usepackage{amsthm} %Para definir ambientes con \newtheorem
\usepackage{amsfonts}
\usepackage{amssymb}
\usepackage{makeidx}
\usepackage{graphicx}
\usepackage[square,sort,comma,numbers]{natbib}
\usepackage{url}
\usepackage{enumitem}
\usepackage{booktabs}

\usepackage{caption} % To make fonts on figure smaller
\captionsetup[figure]{font=small}
\captionsetup[table]{font=small}


%opening
\title{Módulo 2: Estadística y probabilidad con python \newline Probabilidad}
\author{David R. Montalván Hernández}
\date{}

%=========Define los ambientes a utilizar =======%
%Define estilo para dar un salto de línea en el encabezado
%del 'teorema'
\newtheoremstyle{break}
{2ex} %above space
{2ex} %below space
{} %Body font)
{} %indent amount
{\bfseries} %head font
{:} %post head puncuation
{\newline} %post head space
{}

\theoremstyle{break}
%Definición
\newtheorem{definicion}{Definición}[chapter]

%Teorema
\newtheorem{teorema}{Teorema}[chapter]
\newtheorem*{demostracion}{Demostración}

%Proposición
\newtheorem{proposicion}{Proposición}[chapter]

%Notas importantes
\newtheorem{nota}{Nota}[chapter]

%Ejercicios
\newtheorem{ejercicio}{Ejercicio}[chapter]

%Ejemplos
\newtheorem{ejemplo}{Ejemplo}[chapter]

%Algoritmo (Utiliza el ambiente tabbing)
\theoremstyle{break}
\newtheorem{algoritmo}{Algoritmo}[chapter]
%=================================================%

%=================Macros===================%
\newcommand{\mbb}[1]{$\mathbb{#1}$}
\newcommand{\matdim}[2]{$#1 \times #2$}

\begin{document}
\sloppy %Para justificar correctamente (tiene que ver con \usepackage[none]{hyphenat})
\pagenumbering{Roman}
\maketitle
\renewcommand{\contentsname}{Contenido}
\tableofcontents
\renewcommand{\listfigurename}{Lista de imágenes}
\listoffigures
\renewcommand{\listtablename}{Lista de tablas}
\renewcommand\tablename{Tabla}
\renewcommand{\bibname}{Referencias}
\renewcommand{\figurename}{Figura}
\renewcommand{\chaptername}{Capítulo}
\listoftables

\chapter{Conceptos básicos de probabilidad}
\pagenumbering{arabic}
\label{capitulo:conceptos basicos}
En este capítulo se desarrollan las ideas matemáticas detrás de los espacios de probabilidad. Se define de manera formal un evento y la motivación y definición de una sigma álgebra. Se estudia el paradigma frecuentista de la probabilidad y se establecen algunas propiedades de una medida de probabilidad. Finalmente se estudian los conceptos relacionados a la probabilidad condicional e independencia.

\section{Espacios y medidas de probabilidad}
\label{seccion:espacios y medidas de probabilidad}

\subsection{Sigma álgebras}
\label{seccion:sigma algebras}
%Espacio muestral y ejemplos (prob for ML pag 1)
%Definición de evento (prob for ML pag 2)
%Motivación y definición de sigma álgebra (Hoel pag 6)

Supongamos que se quiere modelar algún fenómeno en particular, por ejemplo, la trayectoria del precio de una acción. Lo primero que uno consideraría, es el conjunto de posibles resultados que este fenómeno puede tener.

\begin{definicion}[Espacio muestral]
\label{definicion: espacio muestral}
Sea $\omega$ un resultado posible de un experimento, al conjunto de todos los posibles resultados de este experimento se le llama \textbf{espacio muestral}. Este espacio usualmente es denotado como $\Omega$.
\end{definicion}

Por ejemplo, el lanzamiento de una moneda $2$ veces, tiene el siguiente espacio muestral:
$$
\Omega = \{\left(Cara,Cara\right), \left(Cara,Cruz\right),\left(Cruz,Cara\right), \left(Cruz,Cruz\right)  \}
$$

El espacio muestral $\Omega$, puede ser un conjunto finito, numerable infinito o no numerable infinito y cada elemento de este conjunto se denota con la letra $\omega$.

\begin{ejercicio}
¿Cuál es  el espacio muestral si se busca modelar el precio de una acción?
\end{ejercicio}

Una vez establecido el conjunto $\Omega$, buscamos crear un conjunto $\mathcal{F}$, que contendrá los eventos a los cuales buscamos asignar probabilidades.

\begin{definicion}[Evento]
Sea $\Omega$ un espacio muestral. Cualquier subconjunto $A \subseteq \Omega$, incluyendo el conjunto vacío $\emptyset$ y el mismo conjunto $\Omega$, es llamado un \textbf{evento}.
\end{definicion}

Para tener una idea intuitiva de los conjuntos que deben de pertenecer a $\mathcal{F}$ consideremos dos eventos $A$ y $B$. Claramente si vamos a hablar de la probabilidad de $A$ y $B$, también tendría sentido hablar de la probabilidad de $A \cup B$ y $A \cap B$. Por lo tanto, requerimos que si $A, B \in \mathcal{F}$, entonces $A \cup B \in \mathcal{F}$ y  $A \cap B \in \mathcal{F}$. Finalmente si vamos a decir algo sobre la probabilidad de $A$, claramente tenemos que ser capaces de decir algo también sobre la probabilidad de $A^{c}$, es decir, si $A \in \mathcal{F}$, entonces requerimos que $A^{c} \in \mathcal{F}$.

De manera formal, tenemos la siguiente definición.

\begin{definicion}[Sigma álgebra]
\label{definicion:sigma algebra}
Una colección, $\mathcal{F}$, no vacía de subconjuntos de un conjunto $\Omega$, se llama una \textbf{sigma álgebra} ($\sigma$-álgebra) de subconjuntos de $\Omega$ si las siguientes propiedades se cumplen:

\begin{enumerate}
\item Si $A \in \mathcal{F}$, entonces $A^{c} \in \mathcal{F}$.

\item Si $A_n \in \mathcal{F}$, $n = 1,2, \ldots$, entonces $\cup_{n=1}^{\infty}A_n \in \mathcal{F}$ y $\cap_{n=1}^{\infty}A_n \in \mathcal{F}$.
\end{enumerate}

\end{definicion}

De esta manera, podemos restringir nuestra atención únicamente a los eventos que pertenecen a una sigma álgebra.

\begin{ejercicio}
Demuestre que si $\mathcal{F}$ es una sigma álgebra y $A_n, B_n$, $n=1,2,\ldots$ son una sucesión de eventos de $\mathcal{F}$, entonces el conjunto
$$
\left(\cap_{n = 1}^{\infty} A_{n}^{c}  \right) \cup \left(\cup_{n = 1}^{\infty} B_{n}^{c}  \right)
$$
pertenece a $\mathcal{F}$.
\end{ejercicio}

\begin{ejercicio}
Demuestre que si $\mathcal{F}$ satisface las propiedades de la definición \ref{definicion:sigma algebra}, entonces $\Omega \in \mathcal{F}$ y $\emptyset \in \mathcal{F}$.
\end{ejercicio}

\begin{ejercicio}
Demuestre que una sigma álgebra es cerrada bajo intersecciones y uniones finitas.
\end{ejercicio}

\subsection{Medidas de probabilidad}
%Definición de probabilidad frecuentista (ML a Bayesian pag 11)
%Definición de medida de probabilidad (Hoel pag 8)
%Propiedades de una medida de probabildiad (Hoel 10)
%Fórmula de inclusión y exclusión (prob for ML pag 4)
A continuación se enuncian dos definiciones de  probabilidad, la primera de ellas se basa fuertemente en la intuición frecuentista, mientras que la segunda está basada en la definición axiomática propuesta por Andrey Kolmogorov (1933)

\begin{definicion}[Probabilidad versión frecuentista]
\label{definicion:probabilidad frecuentista}
Sea $\Omega$ un espacio muestral y $\mathcal{F}$ una $\sigma$-álgebra de subconjuntos de $\Omega$. La probabilidad de un evento $A \in \mathcal{F}$, está dada por el límite
$$
\mathbb{P}(A) = \lim_{n \rightarrow \infty} \dfrac{n_A}{n}
$$
en donde $n$ es el número total de ensayos y $n_A$ es el número de veces que ocurre el evento $A$.
\end{definicion}
En la práctica, la definición \ref{definicion:probabilidad frecuentista} se encuentra limitada debido al hecho de que en la realidad $n$ y $n_A$ son números finitos y por lo tanto, lo que realmente se tiene es la aproximación
$$
\mathbb{P}(A) \approx \lim_{n \rightarrow \infty} \dfrac{n_A}{n}
$$

\begin{ejercicio}
Utilizando la definición \ref{definicion:probabilidad frecuentista}, ¿qué implicaría la probabilidad del evento \textit{La empresa $X$ cae en incumplimiento}, si la empresa $X$ es relativamente joven?
\end{ejercicio}

\begin{definicion}[Probabilidad versión axiomática]
Una medida de probabilidad $\mathbb{P}$, definida sobre elementos de una $\sigma$-álgebra $\mathcal{F}$ de subconjuntos de un conjunto $\Omega$, es una función $\mathbb{P}:\mathcal{F} \rightarrow [0,1]$ que cumple lo siguiente:

\begin{enumerate}
\item $\mathbb{P}(\Omega) = 1$.
\item $\mathbb{P}(A) \geq 0$ para todo $A \in \mathcal{F}$.
\item Si $A_n$, $n=1,2,\ldots$ son eventos mutuamente excluyentes (es decir, se tiene que  $A_i \cap A_j = \emptyset$ para $i \neq j$), entonces 
$$
\mathbb{P}\left(\cup_{n=1}^{\infty}A_n \right) = \sum_{n=1}^{\infty}\mathbb{P}(A_n)
$$
\end{enumerate}
\end{definicion}

\begin{definicion}[Espacio de probabilidad]
\label{definicion:espacio de probabilidad}
Un \textbf{espacio de probabilidad} es una $3$-tupla $\left( \Omega, \mathcal{F}, \mathbb{P} \right)$ en donde $\Omega$ es un conjunto, $\mathcal{F}$ es una sigma álgebra de subconjuntos de $\Omega$ y $\mathbb{P}$ es una medida de probabilidad definida sobre $\mathcal{F}$.
\end{definicion}

\begin{ejercicio}
Suponga que el espacio muestral $\Omega$ contiene únicamente un conjunto de $N$ puntos. Suponga además que para todo $\omega \in \Omega$ se tiene que
$$
\mathbb{P}(\omega) = \dfrac{1}{N}
$$
Si $A$ es un evento tal que $|A| = n$, demuestre que
$$
\mathbb{P}(A) = \dfrac{n}{N}
$$
\end{ejercicio}
\subsection{Probabilidad condicional e independencia}
%Definición de probabilidad condicional
%teorema 1.4 (prob for ML pag 5)
%Eventos mutuamente independientes (caso dos eventos y general Hoel pag 19)
%Ejemplo 8 Hoel pag 20

\chapter{Variables aleatorias}
\label{capitulo:variables aleatorias}
%Definición formal de una variable aleatoria
%Definición de una función de densidad
%Definición de una función de distribución
%Ejemplos de distribuciones
%Funciones de una variable aleatoria
%Independencia entre variables aleatorias
%Momentos de una variable aleatoria
%Fórmula de cambio de variable 
%Desigualdades famosas (prob for ML pag 19)
%Fórmula del jacobiano
%Funciones generadoras
%Distribuciones relacionadas a la distribución normal
%Ley de los grandes número y teorema del límite central
%Distribuciones multivariadas
%Distribuciones condicionales
%Ejemplo curse of dimensionality (prob for ML pag 131)
%Ejercicio esperanza condicional minimiza error cuadrático medio

\section{Caso discreto}
\label{seccion:variables discretas}

\section{Caso continuo}
\label{seccion:variables continuas}



\end{document}
